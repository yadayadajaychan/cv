% template author: Cole Miller
% https://www.overleaf.com/latex/templates/coles-resume-template/qhpynjcvjpcj

% Document class and font size
\documentclass[letterpaper,9pt]{extarticle}

% Packages
\usepackage[utf8]{inputenc} % For input encoding
\usepackage{geometry} % For page margins
\geometry{letterpaper, margin=0.75in} % Set paper size and margins
\usepackage{titlesec} % For section title formatting
\usepackage{enumitem} % For itemized list formatting
\usepackage{hyperref} % For hyperlinks

% Formatting
\setlist{noitemsep} % Removes item separation
\titleformat{\section}{\large\bfseries}{\thesection}{1em}{}[\titlerule] % Section title format
\titlespacing*{\section}{0pt}{\baselineskip}{\baselineskip} % Section title spacing

% Begin document
\begin{document}

% Disable page numbers
%\pagestyle{empty}

% Header
\begin{center}
\textbf{\Large ETHAN CHENG}\\[2pt]
	\href{mailto:ethancheng@ucla.edu}{ethancheng@ucla.edu} | (626) 456-2598 | \href{https://www.linkedin.com/in/ethan-cheng-baa662223}{linkedin.com/in/ethan-cheng-baa662223} | \href{http://www.nijika.org/about/}{nijika.org/about}
\end{center}

% Education Section
\section*{EDUCATION}
\noindent
\textbf{University of California, Los Angeles} \hfill September 2023 | Expected: March 2026\\
B.S. Electrical Engineering \hfill GPA: 3.318\\ % TODO GPA

\noindent
\textbf{Skyline College}, San Mateo, CA \hfill August 2022 | May 2023\\
\null \hfill GPA: 3.94

% Experience Section
\section*{EXPERIENCE}

\noindent
\textbf{California NanoSystems Institute (CNSI)} (\url{https://cnsi.ucla.edu/}) \hfill Los Angeles, CA\\
\textit{Lab TA for Technology Training Program} \hfill Fall 2025 - Present % Position and duration
\begin{itemize}
	\item Supported student training in a Class 100/1000 cleanroom environment, emphasizing safe tool operation and process discipline.
	\item Provided hands-on guidance across core fabrication modules including photolithography, thin-film deposition, dry/wet etching, metrology, process control, and CAD.
	\item Explained underlying physical and chemical principles behind each process step to bridge lecture concepts with real fabrication workflows.
	\item Assisted students in debugging process issues, interpreting metrology data, and understanding sources of variation and yield loss.
\end{itemize}

\noindent
\textbf{ECE 121DA Semiconductor Processing and Device Design (UCLA)} \hfill Los Angeles, CA\\
\textit{Student} \hfill Winter 2025 - Spring 2025 % Position and duration
\begin{itemize}
	\item Studied end-to-end CMOS fabrication, including oxidation, diffusion, ion implantation, lithography, and metallization.
	\item Designed semiconductor devices with emphasis on tradeoffs between performance, scaling, and manufacturability.
	\item Used TCAD tools to simulate process steps and predict electrical characteristics prior to fabrication.
	\item Fabricated functional MOSFETs starting from blank silicon wafers, performing full process integration and device testing.
\end{itemize}

\noindent
\textbf{Linux Users Group at UCLA} (\url{https://linux.ucla.edu/}) \hfill Los Angeles, CA\\
\textit{President} \hfill Fall 2024 - Spring 2026 % Position and duration
\begin{itemize}
	\item Led a student organization dedicated to open-source software, Linux systems, and technical literacy.
	\item Planned and hosted workshops, talks, and hands-on events covering Linux fundamentals, tooling, and system administration.
	\item Managed club finances, budgeting, and reimbursements while coordinating with UCLA administration.
	\item Fostered an inclusive technical community focused on peer learning and knowledge-sharing.
\end{itemize}

\noindent
\textbf{A+ Accounting and Tax Services} (\url{https://aplusact.com/}) \hfill Daly City, CA\\
\textit{Accountant's Assistant} \hfill August 2021 - May 2022 % Position and duration
\begin{itemize}
	\item Prepared and filed individual and small-business tax returns in compliance with federal and state regulations.
	\item Reconciled client accounts and financial records, identifying discrepancies and ensuring accuracy.
	\item Assisted with document organization and client communication during peak tax season.
\end{itemize}

%% Additional Experience or Volunteer Work
%\noindent
%\textbf{Project or Volunteer Work Name} \hfill City, State\\ % Project or organization name and location
%\textit{Position Title, Volunteer} \hfill Month Year – Month Year % Position and duration
%\begin{itemize}
%	\item Description of responsibilities and achievements at this position. % Responsibilities and achievements
%\end{itemize}
%
%% Club or Organization Experience
%\textbf{Club or Organization Name} \hfill City, State\\ % Club or organization name and location
%\textit{Position Title} \hfill Month Year – Ongoing % Position and duration
%\begin{itemize}
%	\item Description of responsibilities and achievements at this position. % Responsibilities and achievements
%\end{itemize}

% Projects Section
\section*{PROJECTS}

\noindent
\textbf{2048 on an FPGA} \hfill Los Angeles, CA\\
Project for IEEE at UCLA: \url{https://www.ieeebruins.com/} \hfill Fall 2024 | Spring 2025
\begin{itemize}
	\item Collaborated in a team of three to implement the game \textit{2048} entirely in Verilog on an FPGA.
	\item Designed digital logic for game state management, tile movement, and score tracking.
	\item Implemented VGA graphics output, including real-time rendering and display timing control.
	\item Gained experience with hardware debugging, simulation, and synthesis constraints.
\end{itemize}

\noindent
\textbf{X-Y Recorder} \hfill Los Angeles, CA\\
Personal Project: \url{https://www.nijika.org/projects/plotter/} \hfill Summer 2024 % Project link and duration
\begin{itemize}
	\item Repurposed an X--Y recorder to draw arbitrary images from digital input.
	\item Implemented a subset of G-code, parsing motion commands and coordinating axis movement.
	\item Programmed an Arduino to convert G-code instructions into analog control voltages.
	\item Developed intuition for motion control, coordinate transforms, and hardware interfacing.
\end{itemize}

\noindent
\textbf{Teleprinter} \hfill Los Angeles, CA\\
Personal Project: \url{https://www.nijika.org/projects/typewriter/} \hfill Summer 2024 % Project link and duration
\begin{itemize}
	\item Converted an electric typewriter into a serial-controlled teleprinter.
	\item Reverse-engineered the keyboard matrix and soldered a microcontroller to emulate keystrokes.
	\item Implemented keyboard scanning logic and serial communication with a PC.
	\item Enabled arbitrary text sent over serial to be physically printed by the typewriter.
\end{itemize}

\noindent
\textbf{Class Forum} \hfill Los Angeles, CA\\
Class Project for ``CS 35L Software Construction (UCLA)'' \hfill Spring 2024 % Project link and duration
\begin{itemize}
	\item Designed and built a full-stack web application using PostgreSQL, Flask, and React.
	\item Implemented backend REST APIs and managed application infrastructure.
	\item Applied web security best practices including CORS handling, XSS mitigation, HttpOnly cookies, and cryptographic primitives.
	\item Gained experience balancing usability, security, and maintainability.
\end{itemize}

\noindent
\textbf{Path-following Car} \hfill Los Angeles, CA\\ % Project name and location
%\textit{Project Link:} \url{https://www.projectwebsite.com/} \hfill Month Year % Project link and duration
Class Project for ``ECE 3 Introduction to Electrical Engineering (UCLA)'' \hfill Fall 2023
\begin{itemize}
	\item Designed a control system enabling a small car to autonomously follow a marked path.
	\item Interpreted infrared sensor data and transformed readings into a continuous error signal.
	\item Implemented PID control to minimize tracking error and ensure stable motion.
	\item Tuned control parameters empirically to improve responsiveness and robustness.
\end{itemize}

\noindent
\textbf{ZWC} \hfill South San Francisco, CA\\
Personal Project: \url{https://github.com/yadayadajaychan/zwc} \hfill Summer 2023
\begin{itemize}
	\item Developed a program to embed hidden data using zero-width Unicode characters.
	\item Explored UTF-8 encoding and its implications for text processing.
	\item Designed and implemented a custom CRC scheme to detect transmission errors.
	\item Studied the mathematical basis of CRCs, including polynomial division over finite fields.
\end{itemize}

\noindent
\textbf{Solar Boat} \hfill San Mateo, CA\\ % Project name and location
Engineering and Robotics Club at Skyline College \hfill August 2022 | May 2023 % Project link and duration
\begin{itemize}
	\item Worked on a multidisciplinary team to develop motor control systems for a solar-powered boat.
	\item Programmed an Arduino to interface with electronic speed controllers and an LCD display.
	\item Read and interpreted hardware datasheets to understand electrical and timing constraints.
	\item Designed control logic with real-world hardware limitations in mind.
	\item Competed in the California Solar Regatta against other collegiate teams.
\end{itemize}

% Skills Section
%\section*{SKILLS}
%\begin{itemize}
%	%\item \textbf{Relevant Coursework:} Systems and Signals (ECE 102)
%	\item \textbf{Programming:} C/C++, Java, Python, MATLAB, Go
%	\item \textbf{Software:} Linux, Git
%\end{itemize}

\section*{SKILLS}
\begin{itemize}
	\item \textbf{Programming Languages:}
	\begin{itemize}
		\item C/C++ (systems programming, low-level memory management)
		\item Python (scripting, data analysis, automation)
		\item Java (object-oriented design)
		\item Go (concurrent systems, database-backed applications)
		\item MATLAB (numerical analysis, signal processing)
	\end{itemize}

	\item \textbf{Hardware \& Embedded Systems:}
	\begin{itemize}
		\item Verilog (digital logic design, FSMs, FPGA-based systems)
		\item Arduino (microcontroller programming, sensor and actuator interfacing)
		\item Serial communication (UART)
		\item VGA display interfacing
		\item PID control systems
	\end{itemize}

	\item \textbf{Semiconductor Fabrication \& Devices:}
	\begin{itemize}
		\item CMOS process flow
		\item Photolithography
		\item Thin-film deposition
		\item Dry and wet etching
		\item Metrology and process characterization
		\item MOSFET device fabrication and testing
		\item TCAD process and device simulation
	\end{itemize}

	\item \textbf{Software \& Systems:}
	\begin{itemize}
		\item Linux (command-line tools, system configuration)
		\item Git (version control, collaborative workflows)
		\item Shell scripting
		\item Build systems and debugging tools
	\end{itemize}

	\item \textbf{Web \& Application Development:}
	\begin{itemize}
		\item Flask (backend APIs)
		\item React.js (frontend development)
		\item PostgreSQL (relational database design)
		\item Web security fundamentals (CORS, XSS mitigation, authentication, cryptography)
	\end{itemize}

	\item \textbf{Engineering Foundations:}
	\begin{itemize}
		\item Circuit theory (DC/AC analysis, nodal and mesh methods, Thevenin/Norton equivalents),
		\item Analog and digital signal processing (sampling, filtering, frequency-domain analysis),
		\item Linear systems and signals,
		\item Control systems (feedback, stability, PID control),
		\item Electromagnetism (Maxwell’s equations, EM wave propagation, fields and potentials),
		\item Semiconductor physics (band theory, carrier transport, PN junctions, MOS capacitors),
		\item Digital logic and computer architecture,
		\item Reading and applying hardware datasheets,
		\item Debugging across hardware--software boundaries
	\end{itemize}
\end{itemize}

% End document
\end{document}
